\chapter{Conclusion}\label{ch:conclusion}

%Im Schlussteil soll deutlich werden, was innerhalb der Arbeit erreicht wurde. Dazu kann man sich auf die Anfangs beschriebenen Ziele oder Hypothesen beziehen und die wichtigsten Schritte und Erkenntnisse zusammenfassen. Nach den Ergebnissen kann zusätzlich ein Ausblick gegeben werden, wie sich das Problem weiterentwickeln wird, oder welche weiteren Forschungsfragen sich an die Arbeit anknüpfen.

% The conclusion should conclude the paper and is written for the reader who already has read the paper. In other words: most readers have read the paper when they read the conclusion. Again, this sounds obvious but, again, a lot of conclusions do not read like this. It does not make sense to write a conclusion like "we have shown this and that by using this and that method". Well, this is what the reader has just read (and what he may know since he has read the abstract). A proper conclusion should tell the reader what she can or he could do with the newly acquired knowledge. Answer the question "So what?".

In the course of this work, an extensive corpus of German fan fiction was acquired.
Multiple sources were evaluated and with \emph{FF.de} and \emph{AO3} the most suitable ones were chosen.
This corpus consists of 412,923 stories, their chapters and reviews, as well their respective authors and metadata.

While we have already analyzed it with regard to gender representation, it can be used for further research in the field of fan fiction or natural language processing in general.

For analyzing the corpus, we used the pre-trained named entity recognition model \emph{FLAIR} for extracting persons with their occurrences, trained a LSTM model for predicting those persons genders and counted all used pronouns in the process.
The data obtained in this way was subsequently used for further analysis.

Previously, we have stated that men are overrepresented in media such as canons, television shows, and social media conversations and raised the question of whether this also applies to fan fiction.
Although the majority of readers and writers of fan fiction is female, the dominance of male characters and masculine pronoun usages in the corpus is not shifting.


Younger generations have a lower tendency to write stories about male-slash...


% although the majority of readers and writers of fan fiction is female, the trend towards male oriented works is not shifting. ``In the media industry men are observed as the dominating gender in divisions like television shows, news coverage, social media like, like Twitter conversations, and text authors in general (Milli & Bamman, 2016; Bergstrom et al., 2012; Bretthauer et al., 2007; Jia et al., 2015; Garcia et al., 2014). The question now arises as to whether differences can be identified in the area of fan fiction.''

% fan fiction seems to be a community were ppl live out there usually hidden deviations from heteronormativity (opp. of this word?) in the anonimity of the internet
% scheinbar bedarf an solchen geschichten, der von gesellschaftlicher und komerzieller seite nicht oder unzureichend gestillt wird
% even male expected areas are taken by female writers

% future:
% train flair,
% derive user gender from paratexts,
% use user gender for stereotypical analysis as fast 2016