% https://blog.wordvice.com/video-which-verb-tenses-should-i-use-in-a-research-paper/
% The abstract is written for the potentially interested reader. While writing it, keep in mind that most readers read the abstract before they read the paper (sounds obvious, but many abstracts read like the authors did not consider this). The abstract should give an impression of what the paper will be about. Do not use jargon or any abbreviations here. It should be understandable for non-specialists and even for people from fields somehow far away.

In the course of this thesis, an extensive corpus of German fan fictions was acquired.
Fan fictions are fan-made stories that use existing characters and plot elements from media such as literature, movies, or games and alter them as they see fit.
Multiple sources for this kind of stories were evaluated and the most suitable ones, \emph{FanFiktion.de} and \emph{Archive of Our Own}, were selected.
This corpus consists of 412,923 stories, their chapters and reviews, as well their respective authors and metadata.
To analyze the corpus, we used the state-of-the-art named-entity recognition model \emph{FLAIR} to extract story characters with their number of occurrences, trained an LSTM model using \emph{TensorFlow} and \emph{Keras} to predict the genders of these characters, and in the process counted all personal pronouns used.
We can confirm previous findings regarding a dominance of female authors, male character portrayals and erotic narratives across all genres.
The observed use of gender-specific personal pronouns further supports these claims.
In addition, we find that younger authors feature more female characters and write less about otherwise overrepresentative all-male relationships.
