%Der Hauptteil beginnt mit einem kurzen Kapitel zu den Zielen der Arbeit. Bei sehr kurzen Arbeiten, wie einer Seminararbeit, ist dieser Punkt meist mit der Einleitung schon abgedeckt. Darauf folgt der aktuelle Forschungsstand zum Thema. Dabei werden unterschiedliche Ansätze vorgestellt und ihre Vor- und Nachteile aufgezeigt. Je nach Art des Themas fallen die weiteren inhaltlichen Bestandteile unterschiedlich aus. Im Anhang A befinden sich die inhaltlichen Bausteine für eine theoretische, eine konstruktive (Paradigma der „Design Science“) oder eine empirische Arbeit (Paradigma der „Behavioral Science“).

\chapter{Objectives of this Thesis}\label{ch:thesis-overview}

Missing data for possible data analysis of fan-fictional texts for the German language in digital humanities has already been determined.

We therefore want to generate a comprehensive corpus of German fan fiction texts and accompanying metadata.
Specifically, this corpus should contain all stories as well as their chapters, reviews, associated genres, and fandoms.
It should also include the authors of stories and reviews alike, as well as their profiles.
In order to subsequently analyze a distribution of gender portrayals, some preparatory steps must first be taken.

Usually, a story consists of a collection of characters.
These must first be extracted by a suitable named-entity recognition model.
Since the name of a character as such does not tell us anything about its gender, we need to train a prediction model.
In addition to counting and predicting the character names for each story, we also want to count the number of personal pronouns that refer to a particular gender.

The information obtained in this way can then be used to study gender representation, taking into account suitable metadata.

Opportunities offered by fan fiction have already been addressed and will be explored in the following chapter based on previous research in the area.